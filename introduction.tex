\chapter{Introduction}\label{introduction}

% The introduction of your bachelor thesis introduces the research area, the
% research hypothesis, and the scientific contributions of your work.
% A good narrative structure is the one suggested by Simon Peyton Jones
% \cite{peys04:HowToWriteAGoodResearchPaper}:
% %
% \begin{itemize}
% \item describe the problem / research question
% \item motivate why this problem must be solved
% \item demonstrate that a (new) solution is needed
% \item explain the intuition behind your solution
% \item motivate why / how your solution solves the problem (this is technical)
% \item explain how it compares with related work
% \end{itemize}
% %
% Close the introduction with a paragraph in which the content of the next chapters
% is briefly mentioned (one sentence per chapter).

As the purpose of our digital devices continue to expand, the importance of protecting our sensitive data has become more crucial. The pandemic has proven how much we rely on these devices. The value of our digital resources has increased, which makes this an appealing target for exploitation.\\\\ 
A malicious software, or malware, is a software that target our digital resources. The emerging presence of malware have created different types of malware and new attack methods to our computer systems. According to the AV-Test report, in 2021 the number of malware totaled around 1320 million, a 13 times increase of the report in 2012, which was around 100 million \cite{avtest}.\\\\
Most types of modern malware communicate with external servers of the attackers using different network protocols. DNS (Domain Name Server) is a network protocol used frequently to connect to these externel servers \cite{ist}. The malicious external servers have different domain names to make sure to be available to the malwares. An external server having a single domain or a fixed IP address can be blacklisted. This would make the server not accesible anymore. Therefore, external servers create multiple domain names for the malware to connect to. The multiple domain names are created by generating them using a domain generated algorithm (DGA). The malwares are packed with the same DGA that the external server uses, so that it can generate the same domain names. The domain names generated by the servers are registered in advance to secure those domain names. The malware uses Domain Name Server (DNS) services to connect to each generated domain name till it succesfully conncected to the IP address of the malicious external server.\\\\
\pagebreak
\\\\Recently malware that uses DGA are polymorphic. The malware can generate domain names dynamically. One way to do that is by using time information as a seed for the DGA \cite{arntz_2016}. The polymorphic aspect of the malware improves the concealment and robustness of the malware and brings great challenges to DGA-based malware detection. \\\\ 
Hence, there is a solution needed to defend against DGA-based malware. Traditional detection methods use DNS traffic or domain name language characteristics to extract feature out of the DGA malware. After that machine learning is used to analyze the extracted features and complete the identification and classification of DGA domain names. However, it is difficult to determine the DNS traffic and domain name language characteristics of different types of DGA. Specifically, DGA types that use wordlists to generate domains are difficult to differentiate from benign domains. Therefore, detection schemes based on feature extraction and DNS traffic have a high time and bandwidth cost, and the features that are extracted are not flexible. \\\\ 
More comprehensive tactics are necessary to detect malicious domain names and differentiate them from benign domain names. An improved detection method compared with previous approaches is the detection model based on deep learning. The benefit of deep learning is the automatic extraction of the DGA domain features as well as understanding the context of the domain names.\\\\ We propose a nouveau detection model based on deep learning, the Bidirectional Encoder Representations from Transformers (BERT) to detect DGA domains. BERT is a transformer-based machine learning technique which allows for bidirectional training in models. This in contact to previous efforts, which train on a text sequence from left to right or right to left. The second benefit to BERT is that it is already pretrained on different language representation models \cite{DBLP:journals/corr/abs-1810-04805}. \\\\ 
The original English-language BERT base model is pretrained from unlabeled data from the BooksCorpus \cite{BooksCorpus} with 800M words and English Wikipedia with 2500M words \cite{ColBERT}. In our research we will use a distilled version of BERT, called DistilBERT \cite{Sanh2019DistilBERTAD}. This model recuces the size of BERT by 40\%, while still retaining 97 \% of its language understanding capabilities and being 60\% faster. With our DistilBERT model that is pretrained on uncased English words, we are able to detect DGA domains with low-cost, while still surpassing previous detection models.
\pagebreak
\\\\
The paper is structured as follow. In chapter two we will shortly explain what malware is and what kind of malware types there exist. We will describe how botnets use domain generated algorithms to stay online and avoid detection. After that we will describe machine learning, specifically different neural networks techniques, from older neural network techniques to newer ones. We will discuss what kind of problems and shortcomings these older neural networks have compared to the newer ones. Lastly we will introduce a new deep learning model, called the transformer model. As well as unfold the BERT and DistilBERT models that are transformer-based. We will clarify the benefits of these transformer-based models and how it solved several problems of the older neural network techniques. \\\\
In chapter three we will show a detailed implementation of our DistilBERT detection model to detect DGA-based malware domains. Thereafter, show the kind of results our DistilBERT model has produced. Finally, we will compare our results with previous results of DGA detection models.\\\\
In chapter four we will examine previous research done in DGA detection, while looking at their results and their shortcomings.\\\\
In chapter five we will evaluate all our results and discuss the problems our model has and how to improve it. We will conclude by giving any suggestions for future research in this field.