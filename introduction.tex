\chapter{Introduction}\label{introduction}

% The introduction of your bachelor thesis introduces the research area, the
% research hypothesis, and the scientific contributions of your work.
% A good narrative structure is the one suggested by Simon Peyton Jones
% \cite{peys04:HowToWriteAGoodResearchPaper}:
% %
% \begin{itemize}
% \item describe the problem / research question
% \item motivate why this problem must be solved
% \item demonstrate that a (new) solution is needed
% \item explain the intuition behind your solution
% \item motivate why / how your solution solves the problem (this is technical)
% \item explain how it compares with related work
% \end{itemize}
% %
% Close the introduction with a paragraph in which the content of the next chapters
% is briefly mentioned (one sentence per chapter).

As we continue to increase our tasks in the digital world, the importance of
protecting our sensitive data has become more essential. The pandemic proved us how much we rely on it. The value of our digital resources has increased, which makes it a very interesting target for exploitation. \\\\
A malicious software, or malware, is a software that will cause damage to a
computer system. The increase of the number of malware has been going rapidly. According to AV-Test report, in 2021 the number of malware totaled around 1320 million, a 13 times increase of that of 2012, which was around 100 million \cite{avtest}. As the increase of the total amount of malware has increased, malware types and
new attack methods are created and evolving by the day.\\\\
Most types of modern malware communicate with external servers using different network protocols, where DNS(Domain Name Server) services are used most frequently. The number of malicious domain names are also increasingly rapidly each year \cite{ist}. These malware frequently connect to a botnet, a network of computers
running bots under control of the herder. The herder has a C \& C(command and control) server that a infected computer(bot) is connected to and receives commands from.These malware often are using DNS Services to locate the C \& C servers instead of using fixed IP addresses. In particular, these malware uses DGA(domain generated algorithm) to generate domain names to avoid tracking. As a single domain or a fixed IP address can be easily tracked. These malicious domain names will then be put in a blacklist, which would make those domain names not accessible anymore. While malware that uses DGA connect to malicious domain names, which are active for a limited amount of time. This prevents them to being blocked by blacklist-based countermeasures. The domain names generated by the DGA are registered in advance to secure those generated domain names.\\\\
When a domain name is detected and blocked after an attack, the attacker can register another domain name that was generated by the DGA. Recently malware that uses DGA are polymorphic. When this malware is analyzed and the DGA is examined, the malware can generate different domain names dynamically, by using for time information as a seed for the algorithm \cite{arntz_2016}.\\\\
There is a way needed to defend against DGA-based malware. Current defense techniques against DGA-based malware are using classifications and neural networks to classify and identify malicious domains. Algorithms like SVM, Hidden Markov models, but also neural networks like RNN, CNN or LSTM are used to analyze these domains. Nowadays DGA authors generate domain names that are almost identical to benign domains. This is done by using english dictionary or word lists to make the malicious domain name look believable \cite{arntz_2016}. That makes it harder for classifiers and neural networks to distinguish between these malicious domain names and the benign ones. To tackle this, there is a need for a new defense technique to better differentiate the believable malicious domain names from the unharmful domain names. Therefore, we will use a nouveau neural network technique the BERT classifier to better classify and identify malicious domains.\\\\
Now we can formulate our research question:\\ \textbf{How to detect DGA-based malware using the BERT classifier.}\\\\
In chapter two we will explain in short what malware are and what kind of malware types there are. We will also explain about botnets and how it uses domain generated algorithms to keep the botnet running. Then we will talk about machine learning. Specifically about neural networks and the different type of neural networks. We will talk about older neural networks types that were used by other researchers to identify malicious domains. We will also talk about what kind of problems and shortcomings these neural networks have. At last we will talk about transformers and BERT and clarify the benefits and how it solved some problems of the older neural network techniques.\\\\
In chapter three we will give a detail implementation and all technical details of our BERT classifier to detect detect DGA-based malware samples. We will train our BERT classifier DGA detector/analyzer on recent DGA domain names and unharmful domain names. We will compare our results with other research. In chapter four we will discuss previous research done in this field. What their results and findings are. In chapter five we will discuss and evaluate all our results. We will conclude in chapter five by giving any suggestions for future research on this topic.
