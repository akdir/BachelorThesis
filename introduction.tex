\chapter{Introduction}\label{introduction}

% The introduction of your bachelor thesis introduces the research area, the
% research hypothesis, and the scientific contributions of your work.
% A good narrative structure is the one suggested by Simon Peyton Jones
% \cite{peys04:HowToWriteAGoodResearchPaper}:
% %
% \begin{itemize}
% \item describe the problem / research question
% \item motivate why this problem must be solved
% \item demonstrate that a (new) solution is needed
% \item explain the intuition behind your solution
% \item motivate why / how your solution solves the problem (this is technical)
% \item explain how it compares with related work
% \end{itemize}
% %
% Close the introduction with a paragraph in which the content of the next chapters
% is briefly mentioned (one sentence per chapter).

As we increase to do all our tasks in the digital world, the importance of
protecting our sensitive data is essential. The pandemic showed us how much
we rely on this digital world. The value of our digital resources will thus
also increase, making this a very interesting target for exploitation. \\\\
A malicious software, or malware, is a software that will cause damage to a
computer system. The increase of the number of malware has been going rapidly.
According to AV-Test report, in 2021 the number of malware totaled around 1250
million, a 12 times increase of that of 2012, which was around 100 million [1].
As the increase of the total amount of malware has increased, malware types and
new attack methods are created and evolving by the day.\\\\
Most types of modern malware communicate with external servers using different network protocols, where DNS(Domain Name Server) services are used most frequently. The number of malicious domain names are also increasingly rapidly each year [2]. These malware frequently connect to a botnet, a network of computers
running bots under control of the herder. The herder has a C \& C(command and control) server that a infected computer(bot) is connected to and receives commands from.These malware often are using DNS Services to locate the C \& C servers instead
of using fixed IP addresses [3]. In particular, these malware uses DGA(domain generated algorithm) to generate domain names to avoid tracking. As a single domain or a fixed IP address can be easily tracked. These malicious domain names will then be put in a blacklist, which would make those domain names not accessible anymore. While malware that uses DGA connect to malicious domain names, which are active for a limited amount of time. This prevents them to being blocked by blacklist-based countermeasures. The domain names generated by the DGA are registered in advance
to secure those generated domain names.\\\\
When a domain name is detected and blocked after an attack, the attacker can register another domain name that was generated by the DGA. Recently malware that uses DGA are polymorphic, even when this malware is analyzed and the DGA is examined, the malware can generate different domain names dynamically, by using for example time information as a seed for that algorithm [4].\\\\
There is a way needed to defend against DGA-based malware. Current defense techniques against DGA-based malware is using string information of the generated domain names and analyzing these string information. The attackers avoid this kind of defense techniques by creating DGA that is generating domain names that are almost not distinguishable from normal domain names [5]. To tackle this, there is a need for a new defense technique to differentiate the malicious based domain names from the unharmful ones. To solve this we will combine classification techniques and network traffic analysis.\\\\
Thus we can formulate our research question:\\ \textbf{How to DGA-based malware using BERT Transformer classifier.}\\\\
In chapter two we will explain how domain generated algorithms work and how it is implemented in a malware. As well as explaining the tools that are used in our experiment. We will also explain known network activity patterns in recent DGA-based malware and how these activities effect the system.
In chapter three we will give a detail implementation and all technical details of our DGA detector/analyzer that analyzes, classifies and detect DGA-based malware samples. We will test our DGA detector/analyzer on recent DGA-based malware and unharmful programs that use DNS-based network traffic.
In chapter four we will discuss previous research done in this field.
In chapter five we will discuss and evaluate all our results and findings of the experiment.
We will conclude in chapter six and give any suggestions for future research on this topic.
