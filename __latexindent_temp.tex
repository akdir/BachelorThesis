\chapter{Related Work}\label{relatedwork}
% In this chapter you demonstrate that you are sufficiently aware of the
% state-of-art knowledge of the problem domain that you have investigated as
% well as demonstrating that you have found a \emph{new} solution / approach / method.
In this chapter we will discuss some of the previous work that has been done to detect DGA domains. 
There have been two different ways research on DGA domains has been conducted: in an unsupervised or supervised manner.\\\\ 
Chang and Lin \cite{Chang_Lin} proposed a dynamic way to detect botnets DNS traffic monitoring. First the known benign and malicious domain names were filtered in the DNS traffic.  After which the Chinese-Whispers algorithm was applied to the remaining domains to cluster them according to the similarity of the query behaviour. 
Zhou et al. \cite{Zhou2013DGABasedBD} used a passive DNS dataset to record the information of domain access, consisting of 18 features, to detect Fast-Flux domains using random forest algorithm. 
Antonakakis et al. \cite{Antonakakis} with the insight of knowing that  the DGA domains result in NXDomain responses, classified and clustered the domains with Hidden Markov Models (HMM). Although, because the clustering strategy relied on domain names’ structural and lexical features, it was limited to DGA-based C \& C only.\\\\ 
Woodbridge et al. \cite{WoodbridgeAAG16} were first to utilize supervised deep learning for DGA detection. A simple implementation of an LSTM was used for nonspecific DGA analysis. They have shown that their LSTM network outperforms unsupervised learning methods such as character-level HMM and random forest models. However, their LSTM model did not have a high score on suppobox or matsnu, the dictionary DGA families. Their research led to more research in better identifying DGA families using supervised learning methods. 
In a different angle Anderson et al. used a generative Adversarial Network (GAN) to investigate the use of adversarial learning techniques to deceive DGA detection.
Tran et al. \cite{TRAN20182401} presented a novel LSTM.MI algorithm that combined both binary and multiclass classification models to improve the cost-effectiveness of the LSTM. They demonstrated that the LSTM.MI algorithm provides an improvement of at least 7 \% compared to the original LSTM.
Chen et al. \cite{Chen} propose a LSTM Property and Quantity Dependent Optimization (LSTM.PQDO) that dynamically optimizes the resampling proportion of the original number and characteristics of the samples. They achieved a better performance compared with earlier models by overcoming the difficulties of unbalanced datasets.
Highnam et al. created a novel hybrid neural network, called the BIlbo the fibagginfi€ model, that consists of a model that in parallel uses a convolutional neural network (CNN) and a LSTM network for DGA detection. 
